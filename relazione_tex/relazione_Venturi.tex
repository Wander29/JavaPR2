\documentclass[10pt, a4paper]{article}
%-----------------------
%- 	PACKAGES & SETTINGS
%-----------------------
\usepackage[utf8]{inputenc}
\usepackage[italian]{babel}
\usepackage{xcolor}
\usepackage{hyperref}
\hypersetup{
    colorlinks=true,
    filecolor=magenta,      
    urlcolor=blue,
    linkcolor=black
}
\urlstyle{same}
\usepackage{amsmath}
\usepackage{graphicx}
\graphicspath{ {images/} }
 
%-----------------------
%- 	TITLE
%-----------------------
\title{\textbf{Relazione Progetto Java PR 2}}
\author{\textbf{Venturi} Ludovico\\Docente: \href{http://pages.di.unipi.it/levi/}{Francesca Levi}}
\date{UNIPI, Novembre 2019}


%-----------------------
%- 	DOCUMENT
%-----------------------
\begin{document}
%- 	INTRO
\pagenumbering{roman} 
\maketitle
\tableofcontents
\vfill
\begin{figure}[h]
	\centering
	\includegraphics[scale=0.07]{javaLogo}
	\label{fig:0}
\end{figure}

\clearpage

%- 	START DOC
\pagenumbering{arabic} 
\section{Scelte progettuali}
Nella relazione verranno spiegate le scelte progettuali e implementative che sono state prese.

\subsection{Data}
\begin{center}
OVERVIEW: \textit{Data rappresenta un dato sottoforma di un insieme di 3 attributi e alcune operazioni. È una struttura astratta immutable, di dimensione finita e fissa.}
\end{center}
\textit{Data} viene implementata come classe astratta.
Tale scelta deriva dalla volontà di attribuire a tutte le classi che discendono da \textit{Data} delle caratteristiche comuni, ovvero dei metodi già implementati e una struttura implementativa di base:
\begin{center}
	\textit{private String \textbf{dataName};\\
	private String \textbf{content};\\
	private String \textbf{category};\\}
\end{center}
Come da specifica la classe \textit{Data} riporta anche il metodo astratto display che verrà implementato dalle sottoclassi:
\begin{center}
	\textit{public abstract void display();}
\end{center}
\textit{Data} ridefinisce anche \textit{equals()} per permettere la deep equality e mette a disposizione dei getter per accedere ai dati privati in lettura.\\
Rappresenta un \textit{contratto} cui tutte le sottoclassi (= i vari dati) dovranno sottostare, ovvero condivideranno con \textit{Data} la struttura di base, i vari metodi non astratti e dovranno ridefinire il metodo \textit{display()}.
\subsubsection{Ipotesi}
\begin{itemize}
 \item Non ci sono setter poichè si è ipotizzato che \textit{Data} fosse una struttura \textit{immutable}.
 \end{itemize}
 
\begin{footnotesize}
(Nel testo viene riportato «\textit{i dati possono essere
visualizzati dagli amici ma modificati solamente dal proprietario della bacheca}»: ciò è stato interpretato come: "la modifica consiste nell'aggiunta o la rimozione dei dati, non nella  modifica effettiva del contenuto dei dati").
\end{footnotesize}
\subsubsection{MyData}
\textit{MyData} è una sottoclasse di \textit{Data}. Implementa il metodo display{} senza aggiungere altro alla struttura della sopraclasse.
\clearpage
\subsection{Board «E extends Data» }
\subsubsection{Ipotesi}
\begin{itemize}
\item Non sono ammessi elementi \textit{null}
\item Non sono ammessi duplicati di alcun genere
\item Il numero di likes non dipende solamente dal dato ma anche dalla bacheca in cui si trova $\Rightarrow$ \textit{Data} non possiede il contatore dei like: questo si trova nella bacheca, relativamente ad ogni dato
\end{itemize}

\subsubsection{Implementazione 1}
\begin{figure}[h!]
	\centering
	\includegraphics[scale=0.4]{diag1}
	\label{fig:diag1}
	\caption{Struttura generale del progetto con la prima implementazione di Board}
\end{figure}

Non ho riportato la specifica di ogni metodo nel codice di \textit{Board«E extends Data»} poichè risultava troppo confusionario; l'implementazione ha comunque seguito di pari passo la specifica riportata nell'interfaccia \textit{DataBoard«E extends Data»}.

\subsection{Implementazione 2: Board «E extends Data» }



\clearpage
\section{Eseguire il codice}
\subsection{Test ed esempi}
Non saranno verificate \textit{tutte} i casi in cui parametri sono null per ovvie ragioni, così come tutte le eccezioni ripetute, quali i controlli che la categoria esista o che l'amico sia presente nella lista amici; per queste ultime, pur se generabili in differenti metodi, verranno esplicitamente testate solamente una volta ciascuna. In generale più istruzioni su un singolo blocco \textit{try} indicano che l'ultima sarà quella che genera l'eccezione mentre le precedenti sono eseguite con successo.\\
Lista di test effettuati nel \textit{main}:
\begin{itemize}
\item password della bacheca « 8 caratteri
\item get di una bacheca non presente
\item password errata
\item categoria già presente
\item rimozione di una categoria non presente
\item condivisione di una stessa categoria con uno stesso amico
\item condivisione di una categoria non presente nella bacheca
\item rimozione di un amico non presente nella lista amici
\item rimozione di un amico da una categoria cui non ha accesso, anche se presente nella lista amici
\item inserimento di un dato già presente
\item inserimento di un dato la cui categoria non è presente in bacheca
\item get di un dato la cui categoria non è presente
\item get di un dato non presente
\item get di un dato precedentemente inserito in modo corretto ma la cui categoria è stato poi rimossa
\item rimozione di un dato non presente
\item getDataCategory e modifica della lista ritornata
\item amico vuole inserire like ad un dato di una categoria non condivisa con lui
\item amico vuole inserire like ad un dato cui lo ha già messo
\item amico vuole inserire like ad un dato non presente
\item ITERATORI, prove varie
\item elimino una categoria e itero sui dati di tale categoria tramite una amico con cui essa era condivisa
\end{itemize}
% \href{www.multiplayer.it}{MULTIPLAYER}
% figura \ref{fig:im2} \pageref{fig:im2}
 
%\begin{itemize}
%  \item The individual entries are indicated %with a black dot, a so-called bullet.
%  \item The text in the entries may be of any %length.
%\end{itemize}

%$\begin{enumerate}



%\begin{equation}
%E=mc^2
%\end{equation}
%\subsection{daie}
%\label{sec:daie1}
%$ E=mc^2 $
%$\Omega + 3 = 54$\\
%$\omega * 54$

%\ref{table:tab1} SEZIONE \ref{sec:daie1}

%h = here
%\begin{table}[h]
%\centering
%\begin{tabular}{|c|c|r|}
%	\hline
%		cell1 & gatto & gattini \\
%	\hline
%		cell1 & gatto & 12 \\
%	\hline
%		cell1 & gatto & gattini \\
%	\hline
%\end{tabular}
%\label{table:tab1}
%\end{table}


\end{document}
